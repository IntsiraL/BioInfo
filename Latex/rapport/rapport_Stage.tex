\documentclass[a4paper,10pt]{article}
\usepackage[utf8]{inputenc}
\usepackage[T1]{fontenc}
\usepackage[french]{babel}
\usepackage{graphicx}
\usepackage{amsmath}

\newcommand\tabA[1][0.5cm]{\hspace*{#1}}
\newcommand\tabB[1][1.5cm]{\hspace*{#1}}
\newcommand\tabC[1][2cm]{\hspace*{#1}}
\setlength{\parindent}{0cm}
\setlength{\parskip}{1ex plus 0.5ex minus 0.2ex}
\newcommand{\hsp}{\hspace{20pt}}
\newcommand{\HRule}{\rule{\linewidth}{0.5mm}}
% Title Page
\begin{document}
\begin{titlepage}
  \begin{sffamily}
  \begin{center}

    % Upper part of the page. The '~' is needed because \\
    % only works if a paragraph has started.
    \includegraphics{./image/ua_v_couleur.jpg}
    
    \textsc{ UNIVERSIT\'{E} D'ANGERS \\ UFR INFORMATIQUE}\\[0.5cm]
    \textsc{ \\ Rapport de stage \\ Master 1 2016-2017 }\\[1.5cm]
    
    \includegraphics{./image/logo-generique-SD.png}
    
    \textsc{UMR INSERM 1232 \\-Equipe Immunité Innée et Immunothérapie}\\[1cm]
     
    % Title
    \HRule \\[0.4cm]
    { \huge \bfseries ANALYSE TRANSCRIPTOMIQUE\\[0.4cm] }

    \HRule \\[2cm]

    % Author and supervisor
    \begin{minipage}{0.4\textwidth}
      \begin{flushleft} \large
        \large Présenté par : \\\textsc{RASOLONIAINA Marlino}
      \end{flushleft}
    \end{minipage}
    \begin{minipage}{0.4\textwidth}
      \begin{flushright} \large
        \emph{Tuteur :} M. Le \textsc{Tuteur}\\
        \emph{Chef d'équipe : } M. Chef \textsc{D’Équipe}
      \end{flushright}
    \end{minipage}

    \vfill

    % Bottom of the page
    {\large 12 Avril 2017 — 20 Juin 2017}

  \end{center}
  \end{sffamily}
\end{titlepage}
\newpage
\tableofcontents
\newpage
\section{INTRODUCTION :}
\section{ACQUISITION DES DONN\'{E}ES :}
\subsection{Les puces à ADN :}
  Une puce à ADN est constituée d'un support physique (le plus
  souvent une lame de verre) sur lequel sont déposées des molécules
  d'ADN correspondant à de petits fragments du génome (jusqu'à 40 000
  dépôts différents par puce). On recouvre la puce
  de la solution contenant la population d'ARN à étudier. Les ARN
  s'hybrident sur les fragments d'ADN complémentaires. La quantité d'ARN
  fixée reflète la concentration de cet ARN dans la solution.
  \newline
 Pour des raisons pratiques, on utilise des ADNc plutôt que directement
  les ARN. Les ADNc sont marqués par un nucléotide radioactif ou
  un fluorochrome. Il est possible d'étudier simultanément plusieurs
  populations d'ADNc sur une même puce en utilisant des fluorochromes différents.
  La meilleure façon d'utiliser cette possibilité est de marquer
  de l'ADN génomique avec un fluorochrome, toujours le même. On
  obtient ainsi une référence stable au cours des années
  qui permet de mettre toutes les puces à la même échelle,
  quelle que soit leur origine. 
  \newline
  Un scanner mesure l'intensité du signal émis par l'ADNc hybridé au niveau de chaque dépôt. Parmi les valeurs que proposent les
  logiciels pour cette intensité, la plus fiable est la médiane de l'intensité des pixels car elle est moins sensible aux défauts de
  l'image (pixels sur-brillants par exemple). 
  \newline
  Les puces comportent généralement plusieurs dépôts identiques pour chaque gène. Cela simplifie le travail lorsqu'il faut
  repérer les aberrations dans la lecture des intensités puisqu'il suffit d'examiner les cas où les valeurs diffèrent beaucoup d'un
  dépôt à l'autre. Il s'agit le plus souvent d'un défaut physique sur la puce et il est facile d'éliminer la valeur aberrante.
  Dans le doute, on conserve la médiane des différentes mesures. 
  \newline
  Plusieurs  types  de  puces  à  ADN  
existent selon le support, la nature des fragments fixés à la surface, le mode de fabrication, la 
densité, le mode de marquage des cibles et les méthodes d’hybridation.
\begin{center}
 \includegraphics[scale=0.5]{./image/principe.jpg}
 % Etapes_d'une_expérience_de_biopuces.png: 0x0 pixel, 300dpi, 0.00x0.00 cm, bb=
\end{center}

\subsection{La technologie Illumina :}
Illumina, Inc. est une société américaine qui fabrique et commercialise des systèmes intégrés pour l'analyse de la variation génétique et la fonction biologique 
notamment des gammes de produits et services qui servent les marchés du séquençage, génotypage et expression génétique.
\newline
Un de ces récentes fabrications, le puce ``BeadArray technologie''.
\begin{center}
 \includegraphics[scale=0.5]{./image/beadarray.png}
 % beadarray_multi_sample_array_formats_lg.gif: 0x0 pixel, 300dpi, 0.00x0.00 cm, bb=
\end{center}

\subsection{ Plan expérimental :}
\begin{center}
 \includegraphics[scale=0.5]{./image/plan.png}
 % plan.png: 0x0 pixel, 300dpi, 0.00x0.00 cm, bb=
\end{center}
L’expression des gènes des macrophages a été analysée par des puces d’expression génique de technologie Illumina avec 48210 sondes pour chaque prélèvement.
Ce qui nous donne une matrice de données de dimmension (48210 x 24) avec les 24 prélèvements.
\[
\begin{pmatrix}

   a_{11} & \cdots & a_{1m} \\

   \vdots & \ddots &\vdots \\

   a_{n1} & \cdots & a_{nm} 

\end{pmatrix}
\]
\subsection{Les biais possibles :}
 Il peut exister des biais systématiques dus à d'autres facteurs,
  tels que l'affinité des séquences ou l'efficacité du marquage. 
\section{DESCRIPTION DES DONN\'{E}ES :}
\subsection{ Decryptage et lecture :}
\begin{itemize}
 \item HumanHT-12\_V4\_0\_R2\_15002873\_B.bgx
 \item idat
 \item bgx
 \item sdf
 \item cfg
 \item Metrics.txt
\end{itemize}
\subsection{ Données brutes :}
\begin{itemize}
 \item  Barcode  
\item  Section 
\item  ChipType 
\item  RunInfo 
\item  Quants
      \begin{description}
        \item  MeanBinData        
	\item  TrimmedMeanBinData  
	\item  DevBinData           
	\item  MedianBinData      
	\item  BackgroundBinData   
	\item  BackgroundDevBinData
	\item  CodesBinData      
	\item  NumBeadsBinData    
	\item  NumGoodBeadsBinData
	\item  IllumicodeBinData
      \end{description}
\end{itemize}
\subsection{ Contrôle et qualité :}

\section{PR\'{E}TRAITEMENT DES DONN\'{E}ES :}
\subsection{Transformation :}
\subsection{Normalisation :}
\subsection{Filtrage :}

\section{ANALYSE DES DONN\'{E}ES DE TRANSCRIPTOME :}
\subsection{Gènes différentiellement exprimés :}
\subsection{Gènes co-exprimés :}

\section{INTERP\'{E}TATION :}
Caractérisation d’un ensemble de gènes
\begin{abstract}
\end{abstract}
\end{document}          
